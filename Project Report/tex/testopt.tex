\documentclass[Report.tex]{subfiles}
\begin{document}

\chapter{Testing and Optimisation}
Unit testing and manual testing methods were employed throughout development to ensure that each function continued to return results as expected. An example of a typical unit test is outlined in this chapter, with further elaboration on the methodology employed. A large part of this project concerned the retrieval and manipulation of data from various sources. Refining the input to the APIs and databases resulted in higher success rates of the queries, the creation of more informative datasets, and improved representation of data on the client side. The process of improving these aspects was iterative, and each modification was tested to assess whether the change in output had a positive effect. The key changes in geocoding, querying the MySQL database for concepts, and ordering the concept are discussed in detail in the current chapter.
\section{Testing}
\subsection{Unit Testing}
how does unittest.mock work?
\subsection{Manual Testing}
\section{Optimisation}
\subsection{Geocoding}
\subsection{Querying MySQL}
\subsection{Organising Hierarchical Data}
\end{document}