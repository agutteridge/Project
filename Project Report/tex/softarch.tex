\documentclass[Report.tex]{subfiles}
\begin{document}

\chapter{Software Architecture}
\section{Overview}
One of the main challenges of the project was to integrate the various programming languages and technologies required along the pipeline, that are each suited to their specific task but not immediately compatible with one another. Figure X outlines the flow of the application and the key technologies used for each stage. The details of each component will be explored in the next section, Implementation.

\begin{enumerate}
\item{\textbf{Interaction of the user with the browser}} 
\newline The client side of the web application is presented in a modern browser that can support the use of Scalable Vector Graphics (SVG) that are integral to this D3 application. Examples of this are as Mozilla Firefox 38+, Google Chrome 31+, or Internet Explorer 9+ (although manual scaling is required with the latter (CanIUse.com, \url{http://caniuse.com/#feat=svg}). A text form (!!! is it a form?) is available upon loading of the web page for the user to enter their search term into, SLIDER FOR DATES?
\item{\textbf{Retrieving data from PubMed}}
\newline The PubMed IDs (PMID) of up to 20 papers are retrieved from PubMed via a BioPython function\cite{biopython}. First, a MongoDB cache is queried for any documents with the same PMIDs, which can be retrieved with their MetaMap concepts and Google place IDs. Otherwise, the entry is fetched from PubMed using the unique PMID as a reference. 
\item{\textbf{Extracting medical concepts from paper information}}
\newline Three fields of a standard PubMed entry are utilised in order to extract useful, knowledge-associated terms for visualisation of the topics that the papers cover. These are the abstract, MeSH headings, and keywords as added by the authors. These are submitted together for use by the Semantic Knowledge Representation (SKR) Java Web API which returns a list of formatted concepts and their Concept Unique Identifiers (CUI).
\item{\textbf{Assigning semantic hierarchy to concepts}}
\newline A local copy of the MetaThesaurus set of vocabularies, organised into a database subset compatible with MySQL, is queried for the CUIs as found in the previous step. There is a 1:1 relationship between each CUI and one Semantic Type, which is returned to inform the visualisation of gross hierarchies. To achieve a more granular representation, a direct parent is also retrieved.
\item{\textbf{Finding locations for affiliate addresses}}
\newline Addresses as listed in the 'affiliation information' section of most papers are formatted to improve success rates, and then sent to the Google Places API to retrieve a result detailing the geographical coordinates of the most likely result.
\item{\textbf{Data visualisation}}
\newline Data gained from the previous four stages are combined, primarily to serve as JSON-formatted data to the client, but also to add to the cache for future usage with significantly reduced overhead. The Google Map and D3 canvas are updated to reflect the semantic and geographical information represented the data.
\end{enumerate}

\end{document}

FOR IMPLEMENTATION:
Explain more about what an SVG is.

Research carried out in the planning stage of the project suggested that users are often looking for a paper they are already familiar with, and so another field is available for searching the author fields only. Increasing the specificity of this query with a simple interface allows the user to take full advantage of the additional functionality, whilst the search string is converted to an 'advanced' query server-side.

From here, a loading GIF is served to indicate to the user that the app is functional, and that results are incoming. this is important to improve the user's experience, and to prevent confusion at potentially long wait times.

Why metamap or MTI not all papers have MeSH terms or keywords! (can I find out how many, which vocabularies 100 random papers or from a broad topic e.g. biology? I could do a venn diagram for geocode formataddresses and also show how many have keywords/mesh headings