\documentclass[Report.tex]{subfiles}
\begin{document}

\chapter{Conclusion}
\section{Summary}
By utilising a combination of natural language processing technologies and pre-defined hierarchical vocabularies, this project aimed to visualise information found in biomedical citations that is otherwise presented as text. Without the appropriate analyses, these data are computationally unrelated to real-world structures, despite being intrinsically based upon tangible people (authors), geographical locations (institutions) and topics of interest. MetaMap, as part of the Medical Text Indexer, is used to discover concepts present in the citation, that are then contextualised within the Unified Medical Language System by searching the MetaThesaurus for parent concepts higher up in the tree hierarchy. A graphical representation of concepts is paired with a geographical map of the institutions that the authors are affiliated with, with the intention of facilitating a more immediate understanding of the information associated with the citation.\newline

\noindent These components together comprise the application, allowing a user to interact with a client interface in a web browser, conduct a PubMed search on their chosen query term and then view the concepts and institutions associated with the results displayed using modern JavaScript libraries. Server-side, a combination of data retrieval using HTTP requests, SQL queries and direct activation of local processes is required to achieve these data from various sources, the future performance of which is enhanced by caching results in a document store.

\section{Final Thoughts}
Since the outset, this project has had an open-ended structure, as so much of the functionality is dependent upon the data retrieved from numerous external sources. I think that overall the project has achieved its goal in allowing users to explore citations in a more detailed and visual way, as opposed to the default PubMed user interface. Throughout the course of this project I have picked up many new skills for both front- and back-end web development. However, I do think this had an impact on my productivity, so it is likely that I could have achieved more in the same timeframe with familiar technologies. Despite this, it was quick and straightforward to build a web application in Python using the Flask framework, and there were many D3 examples available to learn from. Both of these libraries were well documented, which made it easier to understand and integrate them into my project. On the other hand, documentation from the NIH was fairly piecemeal and difficult to find, which highlighted the importance of detailed documentation for APIs and libraries.\newline

\noindent Going forward, the accuracy of the geocoding process could be improved by pinpointing key elements of the string that would be best for address recognition, such as street names, postcodes or institution names, regardless of the end geocoding service used. I am happy with the implementation of MetaMap, the MTI and the MetaThesaurus, though the current technology is too demanding for deployment to a cheap server. The user interface could have had better interactivity with a more in-depth understanding of D3. All in all, I am pleased that I challenged myself by utilising new technologies, and that I was also able to use skills gained on the MSc such as SQL querying, imperative programming, and general good practice programming such as unit testing.

\end{document}