\documentclass[Report.tex]{subfiles}
\begin{document}

\chapter{Introduction}
\section{Project Background}
\subsection{The Problem}
The last few decades have seen the availability of digital content dramatically increase, as adoption of the World Wide Web and other internet technologies became widespread both at home and in the workplace. Academic publishing was soon to capitalise on this trend, with long-established journals such as Nature creating websites for all aspects of the publishing pipeline, from submission to publication of papers online, in the '90s\cite{nature-history}. This increase in accessibility has contributed toward an exponential growth of literature\cite{hunter-cohen} that can make the search for relevant papers challenging in the present day. PubMed, a freely-available online database for medical texts maintained by the National Center for Biotechnology Information of the National Institutes of Health (NIH) in the United States, is a mainstay for many users in the health and academic sectors, and is accessed by millions of users daily\cite{dogan}. However, the web interface presents a simplistic approach to information retrieval, with minimal analysis of results and a relatively steep learning curve for achieving satisfactory results with text queries alone. This rich source of information combined with a lack of utilities provided the motivation for this project.

\subsection{Related Work}
A great deal of work has been carried out on the analysis and optimisation of PubMed as a service. The proposal discussed a small number of web applications that allow users to view citations in different ways, such as through the generation of static graphs with GoPubMed\cite{gopubmed}, or by creating a social network-like web application for authors with Microsoft Academic Search\cite{mas}. Of note, GoPubMed implements a text recognition algorithm in order to extract terms that are known to belong to a specific hierarchical vocabulary, the Gene Ontology. This definition of text in terms of real-world concepts is known as word-sense disambiguation (WSD). Citations are then associated with these terms, allowing efficient categorisation and therefore retrieval upon the execution of a relevant query. To my knowledge, GoPubMed does not disambiguate terms, which may result in a lower success rate of assigning vocabulary entries to papers. Natural language processing is particularly challenging in biomedical literature, as multiple nomenclatures can exist for the same concept, such as with Latin and colloquial names (humans vs. \emph{Homo Sapiens}), chemical names (water vs. H{\textsubscript{2}}O) and acronyms (Polymerase Chain Reaction vs. PCR). MetaMap, developed by Alan Aronson at the NIH, can be used to map multiple strings to a single concept present in the Unified Medical Language System (UMLS), enabling a higher theoretical success-rate for WSD. MetaMap can be used to categorise documents, enabling users or developers to create filters to narrow down search results. This approach was adopted by Li and Zu\cite{lizu} to develop an automated alternative to human curation of medical literature. The workflow entails initial analysis of PubMed user logs to obtain a set of citations that are then sorted according to relevance of the topic of interest, as determined by the concepts returned by MetaMap. This approach tackles the problem of excessive search results from the ground up: it builds upon past user data to inform future user behaviour. The neatness of this methodology is acknowledged, though there are concerns of creating a search 'bubble' - for example if a citation is missed by MetaMap, it will then be missed by a large number of users due to its exclusion from the topic filter. The filters would require updating to incorporate more recent user logs, but may then exclude unpopular, but potentially relevant, citations. The strategy of this project is to instead provide users with extra information, leaving the filtering behaviour to be carried out manually. 

\subsection{Objectives}
Upon completion, it was envisioned that this application would:

\begin{enumerate}
\item Investigate the utility of citation data retrieved from PubMed, particularly with relation to:
	\begin{itemize}
	\item Semantic keywords mined from the title, abstract, keywords and Medical Subject Headings,
	\item Addresses as listed in the author affiliations,
	\end{itemize}
\item Organise medical/biological concepts into a hierarchy parseable by a client-side library,
\item Display these data in relation to each citation in a web browser, and enable the user to interact with elements of the webpage.
\end{enumerate}

The sections of this report comprise the following: the main features of the application are outlined in Chapter 2, and the project specification is provided in Chapter 3. The organisation of the software architecture is introduced in Chapter 4. The separate components of the application are explored in further detail in Chapter 5, including background information concerning the external services, where possible. Methods of testing and optimisation are presented in Chapter 6, and the report is concluded in Chapter 7, within which the merits, limitations and future developments of the project are discussed.

\end{document}