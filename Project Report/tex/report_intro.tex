\documentclass[Report.tex]{subfiles}
\begin{document}

\chapter{Introduction}
\section{Project Background}
\subsection{The Problem}
In the 1990s, the availability of digital content increased as popularity of the World Wide Web and the internet grew. Academic publishing was not left out of this trend, with long-established journals such as Nature creating websites for all aspects of the pipeline, from submission to publication of papers online\cite{nature-history}. This increase in accessibility, combined with  EIFSDJDSLHGSRLIGH

A good technical report/thesis Introduction does four things:

1.       It introduces the problem and motivation for the study.

    Tell the reader what the topic of the report is.
    Explain why this topic is important or relevant.

2.       It provides a brief summary of previous engineering and/or scientific work on the topic.

    Here you present an overview what is known about the problem.  You would typically cite earlier studies conducted on the same topic and/or at this same site, and in doing so, you should reveal the yawning void in the knowledge that your brilliant research will fill.
    If you are writing a thesis, you?re going to need a full-blown literature review with very specific details of all of the scientific or engineering work done on the topic to date.  This literature review is usually contained in its own chapter, particularly for PhD theses.  In the introduction, just present a brief overview, sufficient to establish the need for your research.

3.       It outlines the purpose and specific objectives of the project.

    These are linked to solving the problem or filling the knowledge gap identified above.
    Often, the specific objectives are listed in point form. Sometimes a numbered list is used.

4.       It provides a ?road map? for the rest of the report.

    This is so that the reader knows what?s coming and sees the logic of your organization.
    Describe (in approximately one sentence each) the contents of each of the report/thesis chapters.

What doesn?t go in your Introduction?

    Never put any results or decisions in the Introduction.  Just because you are writing it last doesn?t mean you should give away the story. After all ? it?s called the ?Introduction? for a reason. ;-)

\end{document}