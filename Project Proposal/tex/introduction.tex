\documentclass[PROP_AGutteridge_CS.tex]{subfiles}
\begin{document}

\chapter{Introduction}
\section{Personal Background}
The author read Biomedical Science at bachelor level and currently works as a laboratory technician in the field of cancer genetics.

\section{Project Overview}
Since the actualisation of the World Wide Web, the rate of the spread of ideas, theories and innovations has increased dramatically. Researchers publish their findings in peer-reviewed academic journals, a format which has become progressively more plentiful due to digitisation. These resources are made accessible to users via online interfaces provided by databases such as DBLP for the Computer Sciences and PubMed for the Biomedical Sciences, which categorise and index articles to enable querying. Searching PubMed for journal articles and reviews published in 1990 returns 408,502 citations, whereas altering the year to 2013 yields 1,133,536\footnote{Both queries were run on the 17th March, 2015}. This wealth of information necessitates sophisticated tools to aid users in attaining meaningful results from these search engines; Dogan \emph{et al.}\cite{dogan} found that more than a third of PubMed queries from a single month returned over 100 results.\\

\noindent There is also an imprecision in terms of how real-world concepts are represented; it is possible for addresses and names to be written in multiple ways, and for multiple authors to have the same name. Thus, there is currently little explicit association of data to places and people, a problem known as name-entity disambiguation (NED)\cite{hoffart}. The aim of this project is to produce a web application that addresses these two key issues in the PubMed database and associated search engine:
\begin{enumerate}
\item the low affinity between data and information related to institutional addresses, author names, and subject keywords, and
\item the inefficiency of information retrieval experienced by the user.
\end{enumerate}
The extent of NED will be ameliorated by combining bibliographic data with geographical and semantic information from services external to PubMed. To improve the user's search experience, a visual and interactive user interface will be created from these newly informative data, allowing the user to explore search results in relation to a topic of interest or the location of authors and institutions. 

\end{document}