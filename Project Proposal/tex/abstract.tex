\documentclass[PROP_AGutteridge_CS.tex]{subfiles}
\begin{document}

\begin{abstract}
The volume of biomedical literature available online is in an upward trend. Though this heightened discourse within the scientific community is of inarguable benefit, correspondingly sophisticated tools are required for discovery of relevant information. The project as outlined in this proposal is a web application that provides a visual interface through which users can explore the citations retrieved from a bibliographic database, thereby focusing their search. Additional information concerning the geographic locations of institutes and semantics of topic keywords will be incorporated into the dataset, informing the categorisation of results. Herein, the various online sources of bibliographic data are discussed, with PubMed and its associated API chosen as the most appropriate resource. The various scenarios of user interaction are explored, in order to rationalise the objectives and select the appropriate technologies for the project. Python web frameworks and JavaScript libraries are compared and contrasted, leading to the proposed usage of Flask and D3, respectively. A development plan is outlined for the project and any additional features, time permitting. 
\end{abstract}

\end{document}