\documentclass[PROP_AGutteridge_CS.tex]{subfiles}

\begin{document}

\chapter{Background}
\section{PubMed: A Bibliographic Database and Search Engine}
\par
For this project I am focusing on PubMed, a search engine maintained by the National Center for Biotechnology Information (NCBI) as part of a suite of health-related databases as available at http://www.ncbi.nlm.nih.gov/. PubMed enables users to search through a large number of entries on the Medical Literature Analysis and Retrieval System Online (MEDLINE) database as well as full-length texts available from PubMed Central. At the time of writing PubMed is comprised of over 24 million citations, covering a broad range of biomedical subjects and resources. Of importance to the implementation of this project, NCBI provides a public API for retrieval of document summaries from PubMed in either XML or JSON format. However, other resources for bibliographic information were considered in the brainstorming stages of the project. \\
\newline
\emph{compare other DBs!}

%Journals are entered into MEDLINE upon passing review by a select committee, giving PubMed the benefit of curated content. \emph{linker sentence here. Do I need to say again that there are lots of papers?} A search on PubMed for journal articles and reviews published in 1990 returns 408,502 results\footnote{Both queries were run on the 17th March, 2015}, whereas the same search for the year 2013 yields 1,133,536. \\
%\newline
%With such a wealth of information available, the task of searching for articles requires increasingly sophisticated tools. Dogan et al (2009) analysed a dataset of users? PubMed queries and found that more than a third returned over 100 results. Clearly there is a need for tools allowing users to focus their search. In table 1 it can be seen that 44\% of queries contained a bibliographic search term, such as the name of an author, a journal, or a specific article. This suggests that many searches are directed, presumably based on pre-existing knowledge of the desired result. Other top search term categories were disorders, genes, proteins, chemicals and drugs, CAN BE FOUND IN MESH TERMS DESCRIBE MESH TERMS GODDAMIT. White et al (White, Dumais and Teevan, 2008) have described the significantly different characteristics of medical professionals? queries compared to a control group non-experts. What did they find?

\end{document}