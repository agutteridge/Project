\documentclass[PROP_AGutteridge_CS.tex]{subfiles}
\begin{document}

\chapter{Objectives}
The aims as outlined in the previous chapter will be achieved by a two-pronged approach that addresses both back- and front-end aspects of the system. NED will be ameliorated by combining bibliographic data with geographical and semantic information from services external to PubMed. To improve the user's search experience, a visual and interactive user interface will be created from these newly informative data, allowing the user to explore search results in relation to a topic of interest or the location of authors and institutions. This system will be available as a web application that is compatible with modern browsers. The key objectives are comprised of the modular components of the app, namely:
\begin{itemize}
\item{Retrieval of PubMed results}
\item{Semantic categorisation of keywords}
\item{Disambiguation of institutional addresses}
\item{Visualisation of results within a web browser}
\end{itemize}

\section{Semantic Analysis Tools} 
\subsection{ARIANA}

\section{Assignment of Locations to Institutions \emph{whuuut}}
address disambiguation hopefully going to be sorted by Google Places! Considered OSM but location information/disambiguation not as good (ref that?)

\section{Web frameworks}

\section{JavaScript Libraries}
\subsection{Why JavaScript?} 
   
 
\end{document}